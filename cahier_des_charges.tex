%        File: cahier_des_charges.tex
%     Created: sam. oct. 22 06:00  2016 C
% Last Change: sam. oct. 22 06:00  2016 C
%
\documentclass[a4paper]{article}
\usepackage[francais]{babel}
\usepackage[T1]{fontenc}
\frenchspacing
\author{KostiTeam}
\title{Cahier des charges}
\begin{document}

\maketitle
\newpage
\tableofcontents
\newpage

\section{Probl\'ematique}
La virtualisation peut \^etre utile dans diverses situations. En effet, virtualiser un environnement/un objet permet de trouver ses caract\'eristiques optimales à moindre coût. De plus, cela permet aussi d'acqu\'erir des comp\'etences sans pour autant disposer de mat\'eriel. Dans le cadre d'\'etudes informatiques, il peut \^etre pratique de simuler des r\'eseaux dans lesquels des machines dialoguent. Marionnet est un exemple de logiciel de simulation de r\'eseaux. Cependant, après l'avoir utiliser un tant soit peu, il s'avère que plusieurs d\'efauts g\^enent son utilisation : 
\begin{itemize}
  \item crashs assez fr\'equents;
  \item impossible de changer les configurations des machines en marche;
  \item processus d'arr\^et des machines trop fastidieux;
  \item interface peu ergonomique;
  \item logiciel trop gourmand en ressources.
\end{itemize}

\\
Face à ces problèmes, nous avons donc d\'ecid\'e de cr\'eer un nouveau simulateur de r\'eseaux.

\end{document}


