\subsection{Scénarios}

\subsubsection{Premier cas}

\noindent\textbf{TITRE}: Créer une machine virtuelle.\\
\textbf{OBJECTIF}: Pouvoir créer une machine pour l’utiliser dans un réseau.\\
\textbf{ACTEURS}: L’utilisateur.\\
\textbf{TYPE}: Scénario nominale.\\
\textbf{PRECONDITIONS}: Il faut avoir lancé le logiciel.\\
\textbf{POSTCONDITIONS}: La machine est créée.\\
\textbf{DESCRIPTIF}:\\ 
\indent1. L’utilisateur clique sur l’icône de la machine.\\
\indent2. L’utilisateur déplace l’icône sur la zone qui représente le réseau.\\
\indent3. L’utilisateur saisit l’adresse IP4 et/ou IP6 de la machine.\\
\indent4. L’utilisateur saisit le masque 4 et/ou 6 de la machine.\\
\indent5. L’utilisateur veut définir des paramètres de routages.\\
\indent6. L’utilisateur saisit l’adresse destination en IP4/IP6.\\	
\indent7. L'utilisateur saisit l'adresse de la passerelle en IP4/IP6\\
\indent8. L’utilisateur saisi l’interface qu’il veut utiliser.\\
\textbf{FLUX ALTERNATIFS:}
\indent5. L'utilisateur ne veut pas définir des paramètres de routage.\\
\indent\indent5a. L'utilisateur laisse les champs vides.\\
\textbf{CAS REFERENCES}: aucun.\\

\subsubsection{Deuxième cas}
\noindent\textbf{TITRE}: Créer une passerelle.\\
\textbf{OBJECTIF}: Pouvoir créer une passerelle et l’utiliser dans le réseau.\\
\textbf{ACTEURS}: L’utilisateur.\\
\textbf{TYPE}: Scénario nominale.\\
\textbf{PRECONDITIONS}: Avoir lancé le logiciel.\\
\textbf{POSTCONDITIONS}: La passerelle est créée.\\
\textbf{DESCRIPTIF}:\\
\indent1. L’utilisateur clique sur l’icône de la passerelle.\\
\indent2. L’utilisateur déplace l’icône sur la zone qui représente le réseau.\\
\indent3. L’utilisateur choisi le nombre d’interfaces qu’il veut.\\
\indent4. L’utilisateur saisi l’adresse IP4 et/ou IP6 de la passerelle pour toutes ses interfaces.\\
\indent5. L’utilisateur saisi le masque 4 et/ou 6 de la passerelle pour toutes ses interfaces.\\
\indent6. L’utilisateur veut définir des paramètres de routages.\\
\indent7. L’utilisateur saisi l’adresse destination en IP4/IP6.\\
\indent8. L’utilisateur saisi l’adresse de la passerelle en IP4/IP6.\\
\indent9. L’utilisateur saisi l’interface qu’il veut utiliser.\\
\textbf{FLUX ALTERNATIFS}:\\
\indent5. L'utilisateur ne veut pas définir des paramètres de routage.\\
\indent\indent5a. L'utilisateur laisse les champs vides.\\
\textbf{CAS REFERENCES}: aucun.\\


\subsubsection{Troisième cas}
\noindent\textbf{TITRE}: Créer un hub.\\
\textbf{OBJECTIF}: Pouvoir créer un hub pour l’utiliser dans un réseau.\\
\textbf{ACTEURS}: L’utilisateur.\\
\textbf{TYPE}: Scénario nominale.\\
\textbf{PRECONDITIONS}: Il faut avoir lancé le logiciel.\\
\textbf{POSTCONDITIONS}: Le hub est créé.\\
\textbf{DESCRIPTIF}: 
\indent1. L’utilisateur clique sur l’icône du hub.\\
\indent2. L’utilisateur déplace l’icône sur la zone qui représente le réseau.\\
\textbf{CAS REFERENCES}: aucun.\\



\subsubsection{Quatrième cas}
\noindent\textbf{TITRE}: Relier une machine à une machine/hub/passerelle.\\
\textbf{OBJECTIF}: Pouvoir relier une machine avec un autre dispositif pour qu’il puisse se « voir» dans le réseau.\\
\textbf{ACTEURS}: L’utilisateur.\\
\textbf{TYPE}: Scénario nominale.\\
\textbf{PRECONDITIONS}: Il faut avoir lancé le logiciel, créée une machine et un autre dispositif.\\
\textbf{POSTCONDITIONS}: La machine est reliée avec l’autre dispositif.\\
\textbf{DESCRIPTIF}: \\
\indent1. L’utilisateur clique sur l’icône du câble.\\
\indent2. L’utilisateur déplace l’icône sur la machine à relié.\\
\indent3. L’utilisateur clique sur le dispositif auquel il veut relier la machine.\\
\textbf{CAS REFERENCES}: aucun.\\

\subsubsection{Cinquième cas}
\noindent\textbf{TITRE}: Allumer un dispositif.\\
\textbf{OBJECTIF}: Pouvoir allumer un dispositif pour qu’il soit visible sur le réseau.\\
\textbf{ACTEURS}: L’utilisateur.\\
\textbf{TYPE}: Scénario nominale.\\
\textbf{PRECONDITIONS}: Il faut avoir lancé le logiciel, et créer un dispositif.\\
\textbf{POSTCONDITIONS}: Le dispositif est allumé.\\
\textbf{DESCRIPTIF}:\\
\indent1. L’utilisateur clique sur l’icône on/off.\\
\textbf{CAS REFERENCES}: aucun.\\
