
\documentclass[twoside]{article}
\usepackage[francais]{babel}
\usepackage[T1]{fontenc}
\frenchspacing
\author{KostiTeam}
\title{Manuel d'utilisation LXC}
\begin{document}

\maketitle
\newpage
\tableofcontents
\newpage

\section{Bien d\'ebuter}
Toutes les commandes ex\'ecut\'ees sont effectu\'ees en root (super-utilisateur).

\subsection{Installer LXC}
\subsubsection{Arch-linux}
\emph{\#pacman -S lxc arch-install-scripts}
\subsubsection{Debian}
\emph{\#apt-get install lxc}

\subsection{Cr\'eer un container}
\emph{\#lxc-create -t download -n name}: cr\'eer un container de nom name en proposant la liste des images d'OS possibles \`a t\'el\'echarger.\\ 

\emph{\#lxc-create -t download -n name -d debian -r jessie -a i386}: cr\'eer un container de nom \emph{name} en t\'el\'echargeant une image de distribution \emph{debian}, de release \emph{jessie} et d'architecture \emph{i386} (32 bits).\footnote{Cr\'eer un container d'architecture 64 bits sur un host 32 bits \textbf{ne fonctionne pas}.}\\

\emph{\#lxc-ls}: obtenir la liste des containers cr\'e\'es --fancy pour plus de d\'etails.\\
  
\subsection{D\'emarrer un container}
\emph{\#lxc-start -n name  -d}: d\'emarrer le container de nom \emph{name} en daemon \emph{(-d)}.\\
Par d\'efaut, aucun compte utilisateur n'est cr\'e\'e. Il faut donc se connecter en root.\\ 

\emph{\#lxc-attach -n name}: se connecter en root sur le container \emph{name}.\\
Une fois un compte utilisateur cr\'e\'e, il est possible de lancer un terminal sur une session.\\

\emph{\#lxc-console -n name -t 0}: ouvrir un \'ecran de login sur le terminal tty0 du container \emph{name}.\footnote{\textbf{BUG:} sur certaines distributions, -t diff\'erent de 0 ne fonctionne pas}\\

\subsection{Configurer un container}
\subsubsection{Fichier de configuration}
Le fichier de configuration d'un container \emph{name} est contenue dans le fichier /var/lib/lxc/\emph{name}/config.


\end{document}

