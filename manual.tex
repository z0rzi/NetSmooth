
\documentclass[twoside]{article}
\usepackage[francais]{babel}
\usepackage[latin1]{inputenc}
\usepackage[T1]{fontenc}
\author{KostiTeam}
\title{Manuel d'utilisation LXC}
\begin{document}

\maketitle

Toutes les commandes ex\'ecut\'ees sont effectu\'ees en root (super-utilisateur).

Cr\'eer un container\\
\#lxc-create -t download -n name: cr\'eer un container de nom name en proposant la liste des images d’OS possibles \`a t\'el\'echarger. 

\#lxc-create -t download -n name -d debian -r jessie -a i386: cr\'eer un container de nom name en t\'el\'echargeant une image de distribution debian de release jessie et d’architecture i386 (32 bits).
  
Important: cr\'eer un container d\’architecture 64 bits sur un host 32 bits ne fonctionne pas.

\#lxc-ls: obtenir la liste des containers cr\'e\'es –fancy pour une liste plus d\'etaill\'ee.
  
D\'emarrer un container

\#lxc-start -n name  -d d\'emarrer le container de nom name en daemon (-d).
\end{document}

