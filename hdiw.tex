\documentclass{article}
\usepackage[english]{babel}
\usepackage[T1]{fontenc}
\frenchspacing
\author{KostiTeam}
\title{LXC, Comment ca marche?}
\begin{document}

\maketitle
\newpage
\tableofcontents
\newpage

\section{La virtualisation}
La virtualisation est le fait de cr\'eer une version virtuelle d'une entit\'e physique, comme un syst\`eme 
d'exploitation par exemple. Nous pouvons prendre comme exemple Virtual Box, qui est un outil de virtualisation 
permettant d'\'emuler un syst\`eme d'exploitation (linux ou autre). Les diff\'erentes ressources de la machine 
h\^ote sont alors partag\'ees et allou\'ees dynamiquement aux diff\'erentes machines virtuelles par des logiciels
appel\'es hyperviseur.\\

LXC est lui aussi un outil de virtualisation permettant de cr\'eer des machines virtuelles, poss\'edant 
chacune leur propre syst\`eme d'exploitation. Ces machines sont appel\'ees containers; \`a la diff\'erence 
de la plupart des autres outils de virtualisations, LXC permet aux machines virtuelles de partager le
m\^eme noyau que la machine h\^ote, ce qui permet une r\'epartition efficace des ressources. Ce partage est 
permis par l'outil Cgroups du noyau, qui permet de limiter, compter et isoler l'utilisation des ressources.\\

Chaque machine virtuelle est isol\'ee, de la m\^eme mani\`ere que l'isolement d'un programme avec "\emph{chroot}"
: un environnement est cr\'e\'e de mani\`ere \`a ce que les machines virtuelles n'aient pas acc\`es au syst\`eme 
d'exploitation de la machine host. En revanche, la machine host, elle, a acc\`es aux machines virtuelles.
Cette isolation entre les machines virtuelles et la machine host, permet de garantir une certaine s\'ecurit\'e.\\


\section{Les containers, leur fonctionnement}
Les machines host, ou, containers (conteneurs en français)  doivent, en un premier temps, \^etre cr\'e\'es 
\`a l'aide de la commande "\emph{lxc-create…}" (voir notices pour plus de pr\'ecisions sur les commandes). De
nombreux syst\`emes d'exploitation seront alors propos\'es, (nous avons choisi de prendre Debian, Jessie,
i386). Par d\'efaut, les machines cr\'e\'ees ne sont pas configur\'ees: elles n'ont pas d'interfaces, elles 
n'ont pas de compilateur, et les utilitaires pr\'einstall\'es sont tr\`es rudimentaire (pas de ping, ifconfig, tcpdump...).\\

Chaque container poss\`ede un fichier de configuration situ\'e \`a l'emplacement suivant: "\emph{/var/lib/lxc/<nom du container>/config}".
Il est possible de configurer de nombreux aspect du container dans ce fichier (par exemple: modifier
les variables d'environnement, ou changer le nom d'h\^ote ("\emph{hostname}") du container. Voir "\emph{man lxc}"
pour en savoir plus). Ce fichier va notamment permettre de param\'etrer la configuration r\'eseau du container
\`a son d\'emarrage (c'est ce qui va nous int\'eresser en priorit\'e avec ce fichier).\\

Les containers se lancent avec la commande "\emph{lxc-start …}"; il est pr\'ef\'erable de les lancer en d\'emons
(en arri\`ere-plan), puis d'y "\emph{attacher}" un terminal avec "\emph{lxc-attach}", afin d'\^etre connect\'e en super
utilisateur (ou root) car, premi\`erement, par d\'efaut, aucun profil d'utilisateur n'est cr\'e\'e sur le container,
et, de plus, cela permet d'avoir l'entier contr\^ole du container afin de, par exemple, modifier son adresse ipv4.\\


\section{Bridges et fonctionnement r\'eseau}
\subsection{fichier config}
Comme expliqu\'e plus haut, les param\`etres r\'eseaux du container peuvent \^etre modifi\'es via le fichier config.
Dans ce fichier, chaque "\emph{block}" permet de d\'efinir une interface. Un block commence toujours par la 
d\'efinition du type de r\'eseau, nous allons donc nous int\'eresser tout d'abord au type de r\'eseau ("\emph{lxc.network.type}").
Plusieurs types de r\'eseaux sont disponibles, mais, nous allons choisir le type veth, qui signifie Virtual 
Ethernet, ce type permet d'\'etablir un lien entre l'interface virtuelle du container et un pont, pr\'ealablement
cr\'e\'e sur la machine host (nous verrons cette liaison plus en d\'etail par la suite).\\

Ce fichier permet aussi d'\'etablir des adresses ipv4 ("\emph{lxc.network.ipv4}"), ipv6 ("\emph{lxc.network.ipv6}"),
et mac ("\emph{lxc.network.hwaddr}"), ainsi que leurs Broadcast. Cela permet de mettre en place le r\'eseau 
virtuel avant de lancer les machines; bien que ces param\`etres puissent \^etre modifi\'es une fois la machine
lanc\'ee \`a l'aide de l'outil "\emph{ifconfig}".\\

Le dernier param\`etre que nous verrons pour ce fichier est le lien ("\emph{lxc.network.link}"); ce param\`etre
va permettre d'indiquer \`a quel bridge nous voulons connecter notre interface.\\


\subsection{Bridges}
Les bridges (ou pont en francais) sont des \'equipements r\'eseaux qui permettent de relier deux (ou plus)
interfaces de mani\`ere compl\`etement transparente: en observant les paquets qui transitent, le pont peut 
connaitre les adresses mac des interfaces, et ainsi, rediriger les paquets. Les ponts peuvent par exemple,
\^etre utilis\'es pour rediriger une connexion Ethernet: une machine se connecte en Ethernet, une seconde 
machine se connecte \`a la premi\`ere, elles \'etablissent un pont entre deux de leurs interfaces (une interface
de la premi\`ere machine, et une interface de la deuxi\`eme machine), et, ainsi, la deuxi\`eme machine peut 
avoir acc\`es \`a internet.\\

\end{document}
